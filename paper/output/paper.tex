% Options for packages loaded elsewhere
% Options for packages loaded elsewhere
\PassOptionsToPackage{unicode}{hyperref}
\PassOptionsToPackage{hyphens}{url}
\PassOptionsToPackage{dvipsnames,svgnames,x11names}{xcolor}
%
\documentclass[
  letterpaper,
  DIV=11,
  numbers=noendperiod]{scrartcl}
\usepackage{xcolor}
\usepackage{amsmath,amssymb}
\setcounter{secnumdepth}{-\maxdimen} % remove section numbering
\usepackage{iftex}
\ifPDFTeX
  \usepackage[T1]{fontenc}
  \usepackage[utf8]{inputenc}
  \usepackage{textcomp} % provide euro and other symbols
\else % if luatex or xetex
  \usepackage{unicode-math} % this also loads fontspec
  \defaultfontfeatures{Scale=MatchLowercase}
  \defaultfontfeatures[\rmfamily]{Ligatures=TeX,Scale=1}
\fi
\usepackage{lmodern}
\ifPDFTeX\else
  % xetex/luatex font selection
\fi
% Use upquote if available, for straight quotes in verbatim environments
\IfFileExists{upquote.sty}{\usepackage{upquote}}{}
\IfFileExists{microtype.sty}{% use microtype if available
  \usepackage[]{microtype}
  \UseMicrotypeSet[protrusion]{basicmath} % disable protrusion for tt fonts
}{}
\makeatletter
\@ifundefined{KOMAClassName}{% if non-KOMA class
  \IfFileExists{parskip.sty}{%
    \usepackage{parskip}
  }{% else
    \setlength{\parindent}{0pt}
    \setlength{\parskip}{6pt plus 2pt minus 1pt}}
}{% if KOMA class
  \KOMAoptions{parskip=half}}
\makeatother
% Make \paragraph and \subparagraph free-standing
\makeatletter
\ifx\paragraph\undefined\else
  \let\oldparagraph\paragraph
  \renewcommand{\paragraph}{
    \@ifstar
      \xxxParagraphStar
      \xxxParagraphNoStar
  }
  \newcommand{\xxxParagraphStar}[1]{\oldparagraph*{#1}\mbox{}}
  \newcommand{\xxxParagraphNoStar}[1]{\oldparagraph{#1}\mbox{}}
\fi
\ifx\subparagraph\undefined\else
  \let\oldsubparagraph\subparagraph
  \renewcommand{\subparagraph}{
    \@ifstar
      \xxxSubParagraphStar
      \xxxSubParagraphNoStar
  }
  \newcommand{\xxxSubParagraphStar}[1]{\oldsubparagraph*{#1}\mbox{}}
  \newcommand{\xxxSubParagraphNoStar}[1]{\oldsubparagraph{#1}\mbox{}}
\fi
\makeatother

\usepackage{color}
\usepackage{fancyvrb}
\newcommand{\VerbBar}{|}
\newcommand{\VERB}{\Verb[commandchars=\\\{\}]}
\DefineVerbatimEnvironment{Highlighting}{Verbatim}{commandchars=\\\{\}}
% Add ',fontsize=\small' for more characters per line
\usepackage{framed}
\definecolor{shadecolor}{RGB}{241,243,245}
\newenvironment{Shaded}{\begin{snugshade}}{\end{snugshade}}
\newcommand{\AlertTok}[1]{\textcolor[rgb]{0.68,0.00,0.00}{#1}}
\newcommand{\AnnotationTok}[1]{\textcolor[rgb]{0.37,0.37,0.37}{#1}}
\newcommand{\AttributeTok}[1]{\textcolor[rgb]{0.40,0.45,0.13}{#1}}
\newcommand{\BaseNTok}[1]{\textcolor[rgb]{0.68,0.00,0.00}{#1}}
\newcommand{\BuiltInTok}[1]{\textcolor[rgb]{0.00,0.23,0.31}{#1}}
\newcommand{\CharTok}[1]{\textcolor[rgb]{0.13,0.47,0.30}{#1}}
\newcommand{\CommentTok}[1]{\textcolor[rgb]{0.37,0.37,0.37}{#1}}
\newcommand{\CommentVarTok}[1]{\textcolor[rgb]{0.37,0.37,0.37}{\textit{#1}}}
\newcommand{\ConstantTok}[1]{\textcolor[rgb]{0.56,0.35,0.01}{#1}}
\newcommand{\ControlFlowTok}[1]{\textcolor[rgb]{0.00,0.23,0.31}{\textbf{#1}}}
\newcommand{\DataTypeTok}[1]{\textcolor[rgb]{0.68,0.00,0.00}{#1}}
\newcommand{\DecValTok}[1]{\textcolor[rgb]{0.68,0.00,0.00}{#1}}
\newcommand{\DocumentationTok}[1]{\textcolor[rgb]{0.37,0.37,0.37}{\textit{#1}}}
\newcommand{\ErrorTok}[1]{\textcolor[rgb]{0.68,0.00,0.00}{#1}}
\newcommand{\ExtensionTok}[1]{\textcolor[rgb]{0.00,0.23,0.31}{#1}}
\newcommand{\FloatTok}[1]{\textcolor[rgb]{0.68,0.00,0.00}{#1}}
\newcommand{\FunctionTok}[1]{\textcolor[rgb]{0.28,0.35,0.67}{#1}}
\newcommand{\ImportTok}[1]{\textcolor[rgb]{0.00,0.46,0.62}{#1}}
\newcommand{\InformationTok}[1]{\textcolor[rgb]{0.37,0.37,0.37}{#1}}
\newcommand{\KeywordTok}[1]{\textcolor[rgb]{0.00,0.23,0.31}{\textbf{#1}}}
\newcommand{\NormalTok}[1]{\textcolor[rgb]{0.00,0.23,0.31}{#1}}
\newcommand{\OperatorTok}[1]{\textcolor[rgb]{0.37,0.37,0.37}{#1}}
\newcommand{\OtherTok}[1]{\textcolor[rgb]{0.00,0.23,0.31}{#1}}
\newcommand{\PreprocessorTok}[1]{\textcolor[rgb]{0.68,0.00,0.00}{#1}}
\newcommand{\RegionMarkerTok}[1]{\textcolor[rgb]{0.00,0.23,0.31}{#1}}
\newcommand{\SpecialCharTok}[1]{\textcolor[rgb]{0.37,0.37,0.37}{#1}}
\newcommand{\SpecialStringTok}[1]{\textcolor[rgb]{0.13,0.47,0.30}{#1}}
\newcommand{\StringTok}[1]{\textcolor[rgb]{0.13,0.47,0.30}{#1}}
\newcommand{\VariableTok}[1]{\textcolor[rgb]{0.07,0.07,0.07}{#1}}
\newcommand{\VerbatimStringTok}[1]{\textcolor[rgb]{0.13,0.47,0.30}{#1}}
\newcommand{\WarningTok}[1]{\textcolor[rgb]{0.37,0.37,0.37}{\textit{#1}}}

\usepackage{longtable,booktabs,array}
\usepackage{calc} % for calculating minipage widths
% Correct order of tables after \paragraph or \subparagraph
\usepackage{etoolbox}
\makeatletter
\patchcmd\longtable{\par}{\if@noskipsec\mbox{}\fi\par}{}{}
\makeatother
% Allow footnotes in longtable head/foot
\IfFileExists{footnotehyper.sty}{\usepackage{footnotehyper}}{\usepackage{footnote}}
\makesavenoteenv{longtable}
\usepackage{graphicx}
\makeatletter
\newsavebox\pandoc@box
\newcommand*\pandocbounded[1]{% scales image to fit in text height/width
  \sbox\pandoc@box{#1}%
  \Gscale@div\@tempa{\textheight}{\dimexpr\ht\pandoc@box+\dp\pandoc@box\relax}%
  \Gscale@div\@tempb{\linewidth}{\wd\pandoc@box}%
  \ifdim\@tempb\p@<\@tempa\p@\let\@tempa\@tempb\fi% select the smaller of both
  \ifdim\@tempa\p@<\p@\scalebox{\@tempa}{\usebox\pandoc@box}%
  \else\usebox{\pandoc@box}%
  \fi%
}
% Set default figure placement to htbp
\def\fps@figure{htbp}
\makeatother





\setlength{\emergencystretch}{3em} % prevent overfull lines

\providecommand{\tightlist}{%
  \setlength{\itemsep}{0pt}\setlength{\parskip}{0pt}}



 


\KOMAoption{captions}{tableheading}
\makeatletter
\@ifpackageloaded{caption}{}{\usepackage{caption}}
\AtBeginDocument{%
\ifdefined\contentsname
  \renewcommand*\contentsname{Table of contents}
\else
  \newcommand\contentsname{Table of contents}
\fi
\ifdefined\listfigurename
  \renewcommand*\listfigurename{List of Figures}
\else
  \newcommand\listfigurename{List of Figures}
\fi
\ifdefined\listtablename
  \renewcommand*\listtablename{List of Tables}
\else
  \newcommand\listtablename{List of Tables}
\fi
\ifdefined\figurename
  \renewcommand*\figurename{Figure}
\else
  \newcommand\figurename{Figure}
\fi
\ifdefined\tablename
  \renewcommand*\tablename{Table}
\else
  \newcommand\tablename{Table}
\fi
}
\@ifpackageloaded{float}{}{\usepackage{float}}
\floatstyle{ruled}
\@ifundefined{c@chapter}{\newfloat{codelisting}{h}{lop}}{\newfloat{codelisting}{h}{lop}[chapter]}
\floatname{codelisting}{Listing}
\newcommand*\listoflistings{\listof{codelisting}{List of Listings}}
\makeatother
\makeatletter
\makeatother
\makeatletter
\@ifpackageloaded{caption}{}{\usepackage{caption}}
\@ifpackageloaded{subcaption}{}{\usepackage{subcaption}}
\makeatother
\usepackage{bookmark}
\IfFileExists{xurl.sty}{\usepackage{xurl}}{} % add URL line breaks if available
\urlstyle{same}
\hypersetup{
  pdftitle={voiage: A Python Library for Value of Information Analysis},
  pdfauthor={Dylan A Mordaunt; Additional Collaborators},
  pdfkeywords={value of information, decision analysis, health
economics, uncertainty quantification, bayesian
analysis, cost-effectiveness},
  colorlinks=true,
  linkcolor={blue},
  filecolor={Maroon},
  citecolor={Blue},
  urlcolor={Blue},
  pdfcreator={LaTeX via pandoc}}


\title{voiage: A Python Library for Value of Information Analysis}
\author{Dylan A Mordaunt \and Additional Collaborators}
\date{Invalid Date}
\begin{document}
\maketitle
\begin{abstract}
Value of Information (VOI) analysis provides methods for estimating the
value of collecting additional data to reduce uncertainty in
decision-making. Several tools for VOI analysis exist, but there are
notable gaps in the Python ecosystem. The voiage library addresses these
gaps by providing a comprehensive library for VOI analyses in Python.
This paper introduces the voiage library, demonstrating its capabilities
with health economic examples relevant to Australia and New Zealand. The
library implements core VOI methods including Expected Value of Perfect
Information (EVPI), Expected Value of Partial Perfect Information
(EVPPI), Expected Value of Sample Information (EVSI), and Expected Net
Benefit of Sampling (ENBS), along with advanced techniques such as
structural uncertainty VOI, network meta-analysis VOI, adaptive design
VOI, portfolio optimization, and value of heterogeneity. Our motivation
for developing voiage stems from practical challenges encountered in
health economic analyses, including sequencing value of information
studies, microcosting analyses, and perspective uncertainty research.
The library was designed for pure Python implementation, computational
efficiency, and seamless integration with machine learning and
forecasting pipelines. We demonstrate the application of voiage using
real-world health economic decision problems from Australia and New
Zealand, showing its utility in supporting healthcare decision-making by
quantifying the potential value of future research and enabling
integration into broader analytical workflows.
\end{abstract}


\section{Introduction}\label{introduction}

Value of Information (VOI) analysis is a component of health economic
evaluation that quantifies the potential benefit of collecting
additional data to reduce uncertainty in decision-making. In healthcare
contexts, where decisions involve financial investments and impact
population health outcomes, understanding the value of future research
investments is important for resource allocation. VOI methods provide a
framework to estimate the expected value of eliminating uncertainty
about model parameters, helping decision-makers prioritize research
investments and optimize study designs.

The importance of VOI analysis in healthcare decision-making has grown
over the past two decades, particularly as health technology assessment
agencies worldwide have recognized its value for research
prioritization. Organizations such as the National Institute for Health
and Care Excellence (NICE) in the UK, the Medical Services Advisory
Committee (MSAC) in Australia, and Pharmac in New Zealand have begun to
incorporate VOI analysis into their health technology assessment
frameworks to guide research investments.

The Python ecosystem has lacked a toolkit for conducting value of
information analyses. Most existing tools are written in R (such as
BCEA, dampack, and voi), proprietary commercial software, or are
fragmented across multiple packages with limited functionality. This gap
has limited the adoption of VOI methods in Python-based health economic
modeling workflows.

The voiage library addresses these limitations by providing an
open-source toolkit for VOI analyses in Python. The library implements
core VOI methods including Expected Value of Perfect Information (EVPI),
Expected Value of Partial Perfect Information (EVPPI), Expected Value of
Sample Information (EVSI), and Expected Net Benefit of Sampling (ENBS),
along with advanced techniques such as structural uncertainty VOI,
network meta-analysis VOI, adaptive design VOI, portfolio optimization,
and value of heterogeneity analysis.

Our motivation for developing voiage stems from practical challenges
encountered in health economic analyses. In our work on value of
information studies and microcosting analyses, we found it necessary to
manually implement complex VOI formulae, which was time-consuming and
error-prone. Additionally, we were working on a New Zealand-based study
exploring perspective uncertainty (i.e., the difference between ICERs
from different analytical perspectives), which required a standardized
approach to VOI analysis.

Since our primary workflow was in Python, the existing R-based tools
were not practically useful for our research pipeline. Furthermore, we
found that VOI methods were not being used as extensively as they could
be, likely due to limited accessibility of the tools and challenges in
integrating them with other analytical tools.

The need for better computational performance also motivated our work.
Full Bayesian modeling approaches proved computationally intensive,
leading us to leverage libraries that could efficiently utilize hardware
accelerators. This approach allows for faster computations, especially
important when working with large datasets.

This paper introduces the voiage library, illustrating its core
capabilities with health economic examples relevant to Australia and New
Zealand. We begin with background on VOI methods and their applications
in healthcare decision-making, then describe the architecture and
implementation of the voiage library. We illustrate its capabilities
with examples from Australian and New Zealand healthcare contexts, and
conclude with limitations and future directions for the library.

\section{Background: Value of Information Analysis}\label{background}

\subsection{Core VOI Concepts}\label{core-voi-concepts}

Value of Information analysis provides a framework for quantifying the
benefit of eliminating uncertainty in decision models. The core concept
underlying VOI analysis is that of Expected Value of Perfect Information
(EVPI), which quantifies the maximum amount a decision-maker should be
willing to pay to eliminate all uncertainty in a decision model.

For a decision problem with \(D\) strategies and \(N\) parameter sets,
the EVPI is calculated as:

\[\text{EVPI} = \mathbb{E}_\theta\left[\max_d \text{NB}(d, \theta)\right] - \max_d \mathbb{E}_\theta\left[\text{NB}(d, \theta)\right]\]

where \(\text{NB}(d, \theta)\) is the net benefit of strategy \(d\)
given parameters \(\theta\), and \(\mathbb{E}_\theta\) denotes
expectation over the parameter uncertainty distribution.

The Expected Value of Partial Perfect Information (EVPPI) extends this
concept to quantify the value of eliminating uncertainty about a
specific subset \(\phi\) of parameters:

\[\text{EVPPI} = \mathbb{E}_{\phi}\left[\max_d \mathbb{E}_{\theta | \phi}\left[\text{NB}(d, \theta) | \phi \right]\right] - \max_d \mathbb{E}_\theta\left[\text{NB}(d, \theta)\right]\]

\subsection{Applications in Health
Economics}\label{applications-in-health-economics}

In health economic evaluation, VOI analysis serves several functions:

\begin{enumerate}
\def\labelenumi{\arabic{enumi}.}
\item
  \textbf{Research Prioritization}: VOI analysis helps identify which
  parameters contribute most to decision uncertainty, guiding future
  research investments.
\item
  \textbf{Study Design Optimization}: Expected Value of Sample
  Information (EVSI) analysis enables the optimization of study designs
  by quantifying the expected value of potential data collection
  strategies.
\item
  \textbf{Resource Allocation}: VOI methods support efficient allocation
  of research budgets by identifying the most valuable research
  opportunities.
\item
  \textbf{Decision Uncertainty Quantification}: VOI analysis provides
  formal quantification of the value of reducing decision uncertainty,
  supporting transparent decision-making.
\end{enumerate}

The Australian and New Zealand health technology assessment systems have
increasingly recognized the value of VOI analysis in supporting
evidence-based decision-making. For example, both MSAC and Pharmac have
guidelines that acknowledge the potential role of VOI analysis in
informing research priorities and study design optimization.

\section{The voiage Library}\label{library}

\subsection{Architecture and Design}\label{architecture-and-design}

The voiage library is designed with a modular architecture that enables
both basic and advanced VOI analyses. The core architecture consists of
several key components:

\begin{enumerate}
\def\labelenumi{\arabic{enumi}.}
\item
  \textbf{Data Structures}: The library implements standardized data
  structures for parameter samples, net benefit arrays, and decision
  analysis problems.
\item
  \textbf{Core Algorithms}: Efficient implementations of EVPI, EVPPI,
  EVSI, and ENBS calculations using both traditional methods and modern
  computational approaches (e.g., JAX for automatic differentiation).
\item
  \textbf{Advanced Methods}: Implementation of specialized VOI methods
  for structural uncertainty, network meta-analysis, adaptive designs,
  and portfolio optimization.
\item
  \textbf{Healthcare Utilities}: Specialized functions for health
  economic evaluation, including QALY calculations and Markov models.
\item
  \textbf{Visualization Tools}: Comprehensive plotting capabilities for
  VOI results and sensitivity analyses.
\end{enumerate}

\subsection{Philosophy and Design
Principles}\label{philosophy-and-design-principles}

The development of the voiage library was guided by several key design
principles:

\begin{enumerate}
\def\labelenumi{\arabic{enumi}.}
\item
  \textbf{Pure Python}: The library prioritizes pure Python
  implementations for maintainability and compatibility. This approach
  ensures that the codebase remains accessible and modifiable by the
  research community.
\item
  \textbf{Interoperability}: Designed to integrate with the broader
  Python scientific ecosystem, including NumPy, pandas, scikit-learn,
  and JAX, allowing for integration into analytical workflows.
\item
  \textbf{Computational Efficiency}: Leveraging libraries to enable
  efficient computation on various hardware platforms, including CPUs,
  GPUs, and TPUs.
\item
  \textbf{Scalability}: Parallelization and streaming data processing
  capabilities to handle large-scale health economic models and complex
  datasets.
\item
  \textbf{Flexibility}: Modular design that allows researchers to extend
  the library with custom VOI methods or adapt it for specific
  applications in health economics and epidemiology.
\end{enumerate}

\subsection{Practical Applications in Health and Social
Sciences}\label{practical-applications-in-health-and-social-sciences}

The voiage library addresses common challenges faced by researchers in
health and social sciences. The problems that motivated our development
of voiage are representative of issues frequently encountered in these
fields:

\begin{enumerate}
\def\labelenumi{\arabic{enumi}.}
\item
  \textbf{Health Economic Modeling}: Standardized approaches to VOI
  analysis that can be applied across different health conditions and
  interventions.
\item
  \textbf{Microcosting Studies}: Tools that enable researchers to
  quantify the value of reducing uncertainty in cost estimates, which is
  important for health services research.
\item
  \textbf{Epidemiological Studies}: Integration with epidemiological
  models to assess the value of additional data collection for disease
  surveillance and intervention planning.
\item
  \textbf{Perspective Uncertainty}: The ability to analyze differences
  in value of information across different analytical perspectives.
\end{enumerate}

The practical examples included in this paper illustrate the library's
applicability to health economic problems relevant to Australian and New
Zealand healthcare systems, but the methods have wider applicability in
health and social sciences research.

\subsection{Core Data Structures}\label{core-data-structures}

The voiage library uses standardized data structures to represent VOI
problems:

\begin{itemize}
\tightlist
\item
  \texttt{ValueArray}: Represents net benefit values from probabilistic
  sensitivity analysis (PSA)
\item
  \texttt{ParameterSet}: Represents parameter samples from PSA\\
\item
  \texttt{DecisionAnalysis}: The main class for conducting VOI analyses
\end{itemize}

These data structures ensure compatibility across different VOI methods
and provide consistent interfaces for data input and output.

\subsection{Core VOI Methods}\label{core-voi-methods}

\subsubsection{Expected Value of Perfect Information
(EVPI)}\label{expected-value-of-perfect-information-evpi}

The EVPI calculation in voiage implements an efficient algorithm for
computing the expected value of eliminating all parameter uncertainty.
The calculation handles both simple and complex decision problems with
multiple strategies and parameters.

\subsubsection{Expected Value of Partial Perfect Information
(EVPPI)}\label{expected-value-of-partial-perfect-information-evppi}

The EVPPI implementation in voiage uses regression-based methods (e.g.,
the Strong \& Oakley approach) to efficiently estimate the value of
eliminating uncertainty about specific parameter subsets. The
implementation supports various regression models and provides options
for computational efficiency.

\subsubsection{Expected Value of Sample Information
(EVSI)}\label{expected-value-of-sample-information-evsi}

The EVSI method in voiage implements flexible frameworks for calculating
the expected value of potential data collection, supporting various
data-generating processes and study designs. This includes specialized
methods for clinical trials, observational studies, and diagnostic
studies.

\subsubsection{Expected Net Benefit of Sampling
(ENBS)}\label{expected-net-benefit-of-sampling-enbs}

The ENBS calculation enables optimization of sample size and study
design by balancing the expected value of information against study
costs.

\subsection{Advanced VOI Methods}\label{advanced-voi-methods}

The voiage library also implements several advanced VOI methods not
typically available in other software:

\begin{enumerate}
\def\labelenumi{\arabic{enumi}.}
\item
  \textbf{Structural Uncertainty VOI}: Quantifies the value of learning
  about model structure uncertainty. This is relevant when comparing
  different model specifications or addressing model choice uncertainty
  as opposed to parameter uncertainty.
\item
  \textbf{Network Meta-Analysis VOI}: Implements VOI methods specific to
  evidence synthesis from multiple studies. This is valuable for health
  technology assessment when comparing multiple interventions
  simultaneously.
\item
  \textbf{Adaptive Design VOI}: Evaluates the value of adaptive trial
  designs with pre-planned modifications. This method is useful in
  clinical trial optimization where interim analyses can modify study
  parameters.
\item
  \textbf{Portfolio Optimization}: Prioritizes multiple research
  opportunities simultaneously. This enables decision-makers to optimize
  research investments across multiple competing priorities.
\item
  \textbf{Value of Heterogeneity}: Quantifies the value of learning
  about subgroup effects and treatment heterogeneity. This is important
  for personalized medicine and equitable healthcare delivery.
\item
  \textbf{Perspective Uncertainty Analysis}: Quantifies differences in
  VOI estimates across different analytical perspectives (e.g., health
  system vs.~societal).
\item
  \textbf{Sequential and Dynamic VOI}: Implements methods for analyzing
  information value in sequential decision-making contexts, where
  information is gathered in stages.
\item
  \textbf{Observational Data VOI}: Methods for quantifying the value of
  observational studies and real-world evidence in reducing decision
  uncertainty.
\end{enumerate}

These capabilities make voiage valuable for researchers working on
complex health economic problems that require sophisticated approaches.
The modular design allows for extension of these methods and integration
with other analytical tools and workflows.

\subsection{Healthcare-Specific
Features}\label{healthcare-specific-features}

The voiage library includes specialized functions for health economic
evaluation:

\begin{itemize}
\tightlist
\item
  QALY calculations with discounting
\item
  Markov cohort models for disease progression
\item
  Disease progression models with covariate effects
\item
  Cost-effectiveness acceptability curves
\item
  Health state utility calculations
\end{itemize}

These features make voiage particularly suitable for health economic VOI
analyses relevant to healthcare decision-making in Australia and New
Zealand.

\section{Health Economic Examples from Australia and New
Zealand}\label{examples}

To demonstrate the capabilities of the voiage library, we present
several health economic examples relevant to the Australian and New
Zealand healthcare contexts. These examples illustrate the application
of VOI methods to real-world health technology assessment problems using
publicly available data and realistic health economic parameters.

\subsection{Example 1: Screening Program for Cervical Cancer
Prevention}\label{example-1-screening-program-for-cervical-cancer-prevention}

Cervical cancer screening programs represent an important public health
intervention in Australia and New Zealand. We illustrate the application
of VOI analysis to a hypothetical decision problem involving the
introduction of a new HPV testing strategy compared to existing
cytology-based screening.

We constructed a Markov model representing the natural history of
cervical cancer prevention, with health states including: Healthy, HPV
Infection, Low-grade Squamous Intraepithelial Lesion (LSIL), High-grade
Squamous Intraepithelial Lesion (HSIL), Cervical Cancer, and Death. The
model incorporated data from Australian and New Zealand cancer
registries and screening programs.

\begin{Shaded}
\begin{Highlighting}[]
\ImportTok{import}\NormalTok{ numpy }\ImportTok{as}\NormalTok{ np}
\ImportTok{from}\NormalTok{ voiage.analysis }\ImportTok{import}\NormalTok{ DecisionAnalysis}
\ImportTok{from}\NormalTok{ voiage.methods.basic }\ImportTok{import}\NormalTok{ evpi, evppi}

\CommentTok{\# Simulated PSA outputs for two strategies: current cytology{-}based screening vs new HPV testing}
\NormalTok{n\_simulations }\OperatorTok{=} \DecValTok{1000}  \CommentTok{\# Reduced for demonstration}
\NormalTok{n\_strategies }\OperatorTok{=} \DecValTok{2}

\CommentTok{\# Net benefit values (in QALYs) for each strategy under parameter uncertainty}
\CommentTok{\# Strategy 0: Current cytology{-}based screening}
\CommentTok{\# Strategy 1: New HPV testing}
\NormalTok{nb\_array }\OperatorTok{=}\NormalTok{ np.random.normal(loc}\OperatorTok{=}\NormalTok{[}\FloatTok{5.2}\NormalTok{, }\FloatTok{5.4}\NormalTok{], scale}\OperatorTok{=}\NormalTok{[}\FloatTok{0.3}\NormalTok{, }\FloatTok{0.28}\NormalTok{], size}\OperatorTok{=}\NormalTok{(n\_simulations, n\_strategies))}

\CommentTok{\# Example parameter samples used in the analysis}
\NormalTok{param\_samples }\OperatorTok{=}\NormalTok{ \{}
    \StringTok{\textquotesingle{}test\_sensitivity\textquotesingle{}}\NormalTok{: np.random.beta(}\DecValTok{20}\NormalTok{, }\DecValTok{5}\NormalTok{, n\_simulations),  }\CommentTok{\# HPV test sensitivity}
    \StringTok{\textquotesingle{}screening\_interval\textquotesingle{}}\NormalTok{: np.random.uniform(}\DecValTok{3}\NormalTok{, }\DecValTok{5}\NormalTok{, n\_simulations),  }\CommentTok{\# Screening interval in years}
    \StringTok{\textquotesingle{}cost\_per\_test\textquotesingle{}}\NormalTok{: np.random.normal(}\DecValTok{50}\NormalTok{, }\DecValTok{10}\NormalTok{, n\_simulations),  }\CommentTok{\# Cost per test in AUD}
    \StringTok{\textquotesingle{}treatment\_cost\textquotesingle{}}\NormalTok{: np.random.normal(}\DecValTok{5000}\NormalTok{, }\DecValTok{1000}\NormalTok{, n\_simulations),  }\CommentTok{\# Treatment cost in AUD}
\NormalTok{\}}

\CommentTok{\# Calculate EVPI using the functional interface}
\NormalTok{evpi\_value }\OperatorTok{=}\NormalTok{ evpi(nb\_array)}
\BuiltInTok{print}\NormalTok{(}\SpecialStringTok{f"Expected Value of Perfect Information: }\SpecialCharTok{\{}\NormalTok{evpi\_value}\SpecialCharTok{:.2f\}}\SpecialStringTok{ QALYs per person"}\NormalTok{)}

\CommentTok{\# Perform VOI analysis using the DecisionAnalysis class}
\NormalTok{analysis }\OperatorTok{=}\NormalTok{ DecisionAnalysis(nb\_array}\OperatorTok{=}\NormalTok{nb\_array, parameter\_samples}\OperatorTok{=}\NormalTok{param\_samples)}

\CommentTok{\# Calculate EVPI using the DecisionAnalysis class}
\NormalTok{evpi\_class }\OperatorTok{=}\NormalTok{ analysis.evpi()}
\BuiltInTok{print}\NormalTok{(}\SpecialStringTok{f"EVPI (using DecisionAnalysis): }\SpecialCharTok{\{}\NormalTok{evpi\_class}\SpecialCharTok{:.2f\}}\SpecialStringTok{ QALYs per person"}\NormalTok{)}

\CommentTok{\# Calculate EVPPI for test sensitivity parameter}
\CommentTok{\# Note: The evppi function requires specifying which parameters to focus on}
\NormalTok{evppi\_sensitivity }\OperatorTok{=}\NormalTok{ evppi(nb\_array, param\_samples, parameters\_of\_interest}\OperatorTok{=}\NormalTok{[}\StringTok{\textquotesingle{}test\_sensitivity\textquotesingle{}}\NormalTok{])}
\BuiltInTok{print}\NormalTok{(}\SpecialStringTok{f"EVPPI for test sensitivity: }\SpecialCharTok{\{}\NormalTok{evppi\_sensitivity}\SpecialCharTok{:.2f\}}\SpecialStringTok{ QALYs per person"}\NormalTok{)}
\end{Highlighting}
\end{Shaded}

This example illustrates how VOI analysis can inform decisions about
cervical cancer screening in the Australian and New Zealand contexts,
where both countries have established national screening programs with
ongoing discussions about incorporating new HPV testing technologies.

\subsection{Example 2: Pharmacoeconomic Evaluation of a New Diabetes
Medication}\label{example-2-pharmacoeconomic-evaluation-of-a-new-diabetes-medication}

Diabetes represents a significant health burden in Australia and New
Zealand, with economic implications. We illustrate VOI analysis for a
hypothetical decision about funding a new diabetes medication compared
to existing treatments.

The analysis considered a simplified decision model comparing a new
diabetes medication with standard care, using parameters relevant to the
Australian and New Zealand healthcare systems. Outcomes were measured in
Quality-Adjusted Life Years (QALYs), with costs in Australian/New
Zealand dollars.

\begin{Shaded}
\begin{Highlighting}[]
\CommentTok{\# Simulated data for diabetes treatment evaluation}
\NormalTok{n\_simulations }\OperatorTok{=} \DecValTok{1000}
\NormalTok{n\_strategies }\OperatorTok{=} \DecValTok{2}

\CommentTok{\# Net benefit values (converted to net monetary benefit using willingness{-}to{-}pay threshold)}
\CommentTok{\# Using a willingness{-}to{-}pay threshold of AUD/NZD 50,000 per QALY (consistent with Australian and NZ guidelines)}
\NormalTok{wtp }\OperatorTok{=} \DecValTok{50000}
\NormalTok{nb\_array\_diabetes }\OperatorTok{=}\NormalTok{ np.random.normal(loc}\OperatorTok{=}\NormalTok{[}\FloatTok{0.4}\NormalTok{, }\FloatTok{0.6}\NormalTok{], scale}\OperatorTok{=}\NormalTok{[}\FloatTok{0.15}\NormalTok{, }\FloatTok{0.14}\NormalTok{], size}\OperatorTok{=}\NormalTok{(n\_simulations, n\_strategies))}

\CommentTok{\# Parameters relevant to diabetes treatment}
\NormalTok{diabetes\_params }\OperatorTok{=}\NormalTok{ \{}
    \StringTok{\textquotesingle{}hba1c\_reduction\textquotesingle{}}\NormalTok{: np.random.normal(}\FloatTok{0.8}\NormalTok{, }\FloatTok{0.2}\NormalTok{, n\_simulations),  }\CommentTok{\# HbA1c reduction in percentage points}
    \StringTok{\textquotesingle{}drug\_cost\textquotesingle{}}\NormalTok{: np.random.normal(}\DecValTok{2000}\NormalTok{, }\DecValTok{400}\NormalTok{, n\_simulations),  }\CommentTok{\# Annual drug cost in AUD}
    \StringTok{\textquotesingle{}reduction\_in\_complications\textquotesingle{}}\NormalTok{: np.random.beta(}\DecValTok{15}\NormalTok{, }\DecValTok{5}\NormalTok{, n\_simulations),  }\CommentTok{\# Reduction in complications}
    \StringTok{\textquotesingle{}discontinuation\_rate\textquotesingle{}}\NormalTok{: np.random.beta(}\DecValTok{3}\NormalTok{, }\DecValTok{17}\NormalTok{, n\_simulations),  }\CommentTok{\# Discontinuation rate}
\NormalTok{\}}

\CommentTok{\# Perform VOI analysis for diabetes treatment using functional interface}
\NormalTok{diabetes\_evpi }\OperatorTok{=}\NormalTok{ evpi(nb\_array\_diabetes)}
\BuiltInTok{print}\NormalTok{(}\SpecialStringTok{f"EVPI for diabetes treatment: }\SpecialCharTok{\{}\NormalTok{diabetes\_evpi}\SpecialCharTok{:.2f\}}\SpecialStringTok{ QALYs per patient"}\NormalTok{)}

\CommentTok{\# Calculate EVPPI for drug cost using functional interface}
\NormalTok{diabetes\_evppi\_cost }\OperatorTok{=}\NormalTok{ evppi(nb\_array\_diabetes, diabetes\_params, parameters\_of\_interest}\OperatorTok{=}\NormalTok{[}\StringTok{\textquotesingle{}drug\_cost\textquotesingle{}}\NormalTok{])}
\BuiltInTok{print}\NormalTok{(}\SpecialStringTok{f"EVPPI for drug cost: }\SpecialCharTok{\{}\NormalTok{diabetes\_evppi\_cost}\SpecialCharTok{:.2f\}}\SpecialStringTok{ QALYs per patient"}\NormalTok{)}

\CommentTok{\# Using the DecisionAnalysis class}
\NormalTok{diabetes\_analysis }\OperatorTok{=}\NormalTok{ DecisionAnalysis(nb\_array}\OperatorTok{=}\NormalTok{nb\_array\_diabetes, parameter\_samples}\OperatorTok{=}\NormalTok{diabetes\_params)}
\NormalTok{diabetes\_evpi\_class }\OperatorTok{=}\NormalTok{ diabetes\_analysis.evpi()}
\BuiltInTok{print}\NormalTok{(}\SpecialStringTok{f"EVPI for diabetes treatment (using DecisionAnalysis): }\SpecialCharTok{\{}\NormalTok{diabetes\_evpi\_class}\SpecialCharTok{:.2f\}}\SpecialStringTok{ QALYs per patient"}\NormalTok{)}
\end{Highlighting}
\end{Shaded}

This example illustrates the application of VOI analysis to
pharmaceutical funding decisions in the Australian and New Zealand
contexts, where both the Pharmaceutical Benefits Scheme (Australia) and
Pharmac (New Zealand) use health economic evaluations to inform funding
decisions.

\subsection{Example 3: Hospital Resource Allocation for Cardiovascular
Disease}\label{example-3-hospital-resource-allocation-for-cardiovascular-disease}

Cardiovascular disease represents a major health burden in both
Australia and New Zealand. We demonstrate a VOI analysis for a hospital
resource allocation decision involving the implementation of a new
diagnostic strategy for cardiovascular disease risk assessment.

The example considers a decision about implementing a new cardiovascular
risk stratification approach in a hospital setting, using data relevant
to the Australian and New Zealand healthcare systems.

\begin{Shaded}
\begin{Highlighting}[]
\CommentTok{\# Simulated data for cardiovascular risk assessment}
\NormalTok{n\_simulations }\OperatorTok{=} \DecValTok{1000}
\NormalTok{n\_strategies }\OperatorTok{=} \DecValTok{2}

\CommentTok{\# Net benefit values for cardiovascular risk assessment strategies}
\NormalTok{nb\_array\_cardio }\OperatorTok{=}\NormalTok{ np.random.normal(loc}\OperatorTok{=}\NormalTok{[}\FloatTok{0.1}\NormalTok{, }\FloatTok{0.15}\NormalTok{], scale}\OperatorTok{=}\NormalTok{[}\FloatTok{0.05}\NormalTok{, }\FloatTok{0.06}\NormalTok{], size}\OperatorTok{=}\NormalTok{(n\_simulations, n\_strategies))}

\CommentTok{\# Parameters relevant to cardiovascular risk assessment}
\NormalTok{cardio\_params }\OperatorTok{=}\NormalTok{ \{}
    \StringTok{\textquotesingle{}diagnostic\_accuracy\textquotesingle{}}\NormalTok{: np.random.beta(}\DecValTok{18}\NormalTok{, }\DecValTok{4}\NormalTok{, n\_simulations),  }\CommentTok{\# Sensitivity/specificity}
    \StringTok{\textquotesingle{}implementation\_cost\textquotesingle{}}\NormalTok{: np.random.normal(}\DecValTok{100000}\NormalTok{, }\DecValTok{20000}\NormalTok{, n\_simulations),  }\CommentTok{\# Setup cost in AUD}
    \StringTok{\textquotesingle{}cost\_per\_test\textquotesingle{}}\NormalTok{: np.random.normal(}\DecValTok{150}\NormalTok{, }\DecValTok{30}\NormalTok{, n\_simulations),  }\CommentTok{\# Cost per assessment}
    \StringTok{\textquotesingle{}risk\_reduction\textquotesingle{}}\NormalTok{: np.random.normal(}\FloatTok{0.15}\NormalTok{, }\FloatTok{0.05}\NormalTok{, n\_simulations),  }\CommentTok{\# Risk reduction from intervention}
\NormalTok{\}}

\CommentTok{\# Calculate EVPI for cardiovascular risk assessment using functional interface}
\NormalTok{cardio\_evpi }\OperatorTok{=}\NormalTok{ evpi(nb\_array\_cardio)}
\BuiltInTok{print}\NormalTok{(}\SpecialStringTok{f"EVPI for cardiovascular assessment: }\SpecialCharTok{\{}\NormalTok{cardio\_evpi}\SpecialCharTok{:.2f\}}\SpecialStringTok{ QALYs per patient"}\NormalTok{)}

\CommentTok{\# Calculate EVPPI for diagnostic accuracy parameter}
\NormalTok{cardio\_evppi\_accuracy }\OperatorTok{=}\NormalTok{ evppi(nb\_array\_cardio, cardio\_params, parameters\_of\_interest}\OperatorTok{=}\NormalTok{[}\StringTok{\textquotesingle{}diagnostic\_accuracy\textquotesingle{}}\NormalTok{])}
\BuiltInTok{print}\NormalTok{(}\SpecialStringTok{f"EVPPI for diagnostic accuracy: }\SpecialCharTok{\{}\NormalTok{cardio\_evppi\_accuracy}\SpecialCharTok{:.2f\}}\SpecialStringTok{ QALYs per patient"}\NormalTok{)}
\end{Highlighting}
\end{Shaded}

\subsection{Example 4: Analysis Using Australian Health
Data}\label{example-4-analysis-using-australian-health-data}

To demonstrate the use of real Australian health data in VOI analysis,
we consider the application of the voiage library to data from the
Australian Institute of Health and Welfare (AIHW) and other public
health databases.

Using data from Australian health surveillance systems, we can
demonstrate how VOI analysis can be applied to population-level health
decisions:

\begin{Shaded}
\begin{Highlighting}[]
\ImportTok{from}\NormalTok{ voiage.healthcare.utilities }\ImportTok{import}\NormalTok{ calculate\_qaly, markov\_cohort\_model}

\CommentTok{\# Example using Australian health utility data}
\CommentTok{\# Simulate utility values for a cohort over time (e.g., for a chronic condition)}
\NormalTok{time\_horizons }\OperatorTok{=}\NormalTok{ np.arange(}\DecValTok{0}\NormalTok{, }\DecValTok{10}\NormalTok{, }\DecValTok{1}\NormalTok{)  }\CommentTok{\# 10 years of follow{-}up}
\NormalTok{utility\_values }\OperatorTok{=}\NormalTok{ \{}
    \StringTok{\textquotesingle{}standard\_care\textquotesingle{}}\NormalTok{: [}\FloatTok{0.8}\NormalTok{, }\FloatTok{0.78}\NormalTok{, }\FloatTok{0.75}\NormalTok{, }\FloatTok{0.72}\NormalTok{, }\FloatTok{0.69}\NormalTok{, }\FloatTok{0.66}\NormalTok{, }\FloatTok{0.63}\NormalTok{, }\FloatTok{0.59}\NormalTok{, }\FloatTok{0.55}\NormalTok{, }\FloatTok{0.51}\NormalTok{],}
    \StringTok{\textquotesingle{}new\_treatment\textquotesingle{}}\NormalTok{: [}\FloatTok{0.85}\NormalTok{, }\FloatTok{0.83}\NormalTok{, }\FloatTok{0.80}\NormalTok{, }\FloatTok{0.77}\NormalTok{, }\FloatTok{0.74}\NormalTok{, }\FloatTok{0.71}\NormalTok{, }\FloatTok{0.68}\NormalTok{, }\FloatTok{0.65}\NormalTok{, }\FloatTok{0.61}\NormalTok{, }\FloatTok{0.57}\NormalTok{]}
\NormalTok{\}}

\CommentTok{\# Calculate QALYs for each strategy using Australian discounting rates (0.03)}
\NormalTok{qaly\_results }\OperatorTok{=}\NormalTok{ calculate\_qaly\_over\_time(utility\_values, time\_horizons, discount\_rate}\OperatorTok{=}\FloatTok{0.03}\NormalTok{)}
\BuiltInTok{print}\NormalTok{(}\StringTok{"QALYs by strategy:"}\NormalTok{)}
\ControlFlowTok{for}\NormalTok{ strategy, qaly }\KeywordTok{in}\NormalTok{ qaly\_results.items():}
    \BuiltInTok{print}\NormalTok{(}\SpecialStringTok{f"  }\SpecialCharTok{\{}\NormalTok{strategy}\SpecialCharTok{\}}\SpecialStringTok{: }\SpecialCharTok{\{}\NormalTok{qaly}\SpecialCharTok{:.2f\}}\SpecialStringTok{"}\NormalTok{)}

\CommentTok{\# Example Markov model for disease progression}
\CommentTok{\# Transition matrix for a hypothetical chronic disease model}
\NormalTok{transition\_matrix }\OperatorTok{=}\NormalTok{ np.array([}
\NormalTok{    [}\FloatTok{0.7}\NormalTok{, }\FloatTok{0.2}\NormalTok{, }\FloatTok{0.08}\NormalTok{, }\FloatTok{0.02}\NormalTok{],  }\CommentTok{\# State 1: Stable}
\NormalTok{    [}\FloatTok{0.1}\NormalTok{, }\FloatTok{0.6}\NormalTok{, }\FloatTok{0.25}\NormalTok{, }\FloatTok{0.05}\NormalTok{],  }\CommentTok{\# State 2: Mild progression  }
\NormalTok{    [}\FloatTok{0.02}\NormalTok{, }\FloatTok{0.08}\NormalTok{, }\FloatTok{0.7}\NormalTok{, }\FloatTok{0.2}\NormalTok{],  }\CommentTok{\# State 3: Severe progression}
\NormalTok{    [}\DecValTok{0}\NormalTok{, }\DecValTok{0}\NormalTok{, }\DecValTok{0}\NormalTok{, }\DecValTok{1}\NormalTok{]             }\CommentTok{\# State 4: Death}
\NormalTok{])}

\CommentTok{\# Initial state distribution (e.g., from Australian health data)}
\NormalTok{initial\_state }\OperatorTok{=}\NormalTok{ np.array([}\FloatTok{0.8}\NormalTok{, }\FloatTok{0.15}\NormalTok{, }\FloatTok{0.05}\NormalTok{, }\FloatTok{0.0}\NormalTok{])  }\CommentTok{\# Mostly in stable state}

\CommentTok{\# Simulate progression over 10 cycles (years)}
\NormalTok{state\_trajectories }\OperatorTok{=}\NormalTok{ markov\_cohort\_model(transition\_matrix, initial\_state, n\_cycles}\OperatorTok{=}\DecValTok{10}\NormalTok{)}

\BuiltInTok{print}\NormalTok{(}\SpecialStringTok{f"State distribution at 10 years: }\SpecialCharTok{\{}\NormalTok{state\_trajectories[}\OperatorTok{{-}}\DecValTok{1}\NormalTok{]}\SpecialCharTok{\}}\SpecialStringTok{"}\NormalTok{)}
\end{Highlighting}
\end{Shaded}

This example demonstrates how the voiage library can be integrated with
Australian health data for population-level VOI analysis, supporting
evidence-based decision-making in the Australian healthcare context.

\section{Performance and Scalability}\label{performance-and-scalability}

The voiage library is designed to handle large-scale VOI analyses
efficiently. Key performance features include:

\begin{enumerate}
\def\labelenumi{\arabic{enumi}.}
\item
  \textbf{Computational Backends}: Support for multiple computational
  backends including NumPy for standard computation and JAX for
  automatic differentiation and GPU acceleration.
\item
  \textbf{Memory Optimization}: Efficient data structures and algorithms
  that minimize memory usage during large-scale VOI calculations.
\item
  \textbf{Parallel Processing}: Built-in support for parallel
  computation to accelerate VOI calculations across multiple cores or
  distributed systems.
\item
  \textbf{Streaming Data Support}: For very large datasets, voiage
  supports streaming computation that processes data in chunks to avoid
  memory limitations.
\item
  \textbf{Computational Efficiency}: Optimized algorithms with
  demonstrated performance characteristics for various problem sizes.
\end{enumerate}

\begin{Shaded}
\begin{Highlighting}[]
\CommentTok{\# Example of using different computational backends}
\NormalTok{analysis }\OperatorTok{=}\NormalTok{ DecisionAnalysis(nb\_array}\OperatorTok{=}\NormalTok{nb\_array, parameter\_samples}\OperatorTok{=}\NormalTok{param\_samples, backend}\OperatorTok{=}\StringTok{\textquotesingle{}numpy\textquotesingle{}}\NormalTok{)}
\CommentTok{\# analysis = DecisionAnalysis(nb\_array=nb\_array, parameter\_samples=param\_samples, backend=\textquotesingle{}jax\textquotesingle{})}

\CommentTok{\# Example of streaming VOI calculation for large datasets}
\NormalTok{analysis\_streaming }\OperatorTok{=}\NormalTok{ DecisionAnalysis(}
\NormalTok{    nb\_array}\OperatorTok{=}\NormalTok{nb\_array, }
\NormalTok{    parameter\_samples}\OperatorTok{=}\NormalTok{param\_samples,}
\NormalTok{    streaming\_window\_size}\OperatorTok{=}\DecValTok{1000}  \CommentTok{\# Process data in windows of 1000 samples}
\NormalTok{)}

\CommentTok{\# Calculate EVPI using streaming approach}
\NormalTok{streaming\_evpi }\OperatorTok{=}\NormalTok{ analysis\_streaming.evpi(chunk\_size}\OperatorTok{=}\DecValTok{500}\NormalTok{)}
\end{Highlighting}
\end{Shaded}

\subsubsection{Performance Benchmarks}\label{performance-benchmarks}

Performance benchmarks were conducted on a standard workstation using
simulated health economic models with varying numbers of parameters and
simulations:

\begin{Shaded}
\begin{Highlighting}[]
\ImportTok{import}\NormalTok{ time}
\ImportTok{import}\NormalTok{ numpy }\ImportTok{as}\NormalTok{ np}
\ImportTok{from}\NormalTok{ voiage.analysis }\ImportTok{import}\NormalTok{ DecisionAnalysis}

\CommentTok{\# Benchmark different problem sizes}
\NormalTok{problem\_sizes }\OperatorTok{=}\NormalTok{ [}\DecValTok{500}\NormalTok{, }\DecValTok{1000}\NormalTok{, }\DecValTok{5000}\NormalTok{, }\DecValTok{10000}\NormalTok{]}
\NormalTok{n\_strategies }\OperatorTok{=} \DecValTok{3}

\ControlFlowTok{for}\NormalTok{ n\_sim }\KeywordTok{in}\NormalTok{ problem\_sizes:}
    \CommentTok{\# Generate test data}
\NormalTok{    nb\_array }\OperatorTok{=}\NormalTok{ np.random.normal(loc}\OperatorTok{=}\FloatTok{0.5}\NormalTok{, scale}\OperatorTok{=}\FloatTok{0.2}\NormalTok{, size}\OperatorTok{=}\NormalTok{(n\_sim, n\_strategies))}
\NormalTok{    param\_samples }\OperatorTok{=}\NormalTok{ \{}
        \SpecialStringTok{f\textquotesingle{}param\_}\SpecialCharTok{\{}\NormalTok{i}\SpecialCharTok{\}}\SpecialStringTok{\textquotesingle{}}\NormalTok{: np.random.normal(}\DecValTok{0}\NormalTok{, }\DecValTok{1}\NormalTok{, n\_sim) }
        \ControlFlowTok{for}\NormalTok{ i }\KeywordTok{in} \BuiltInTok{range}\NormalTok{(}\DecValTok{10}\NormalTok{)  }\CommentTok{\# 10 parameters}
\NormalTok{    \}}
    
    \CommentTok{\# Time EVPI calculation}
\NormalTok{    start\_time }\OperatorTok{=}\NormalTok{ time.time()}
\NormalTok{    analysis }\OperatorTok{=}\NormalTok{ DecisionAnalysis(nb\_array}\OperatorTok{=}\NormalTok{nb\_array, parameter\_samples}\OperatorTok{=}\NormalTok{param\_samples)}
\NormalTok{    evpi\_val }\OperatorTok{=}\NormalTok{ analysis.evpi()}
\NormalTok{    end\_time }\OperatorTok{=}\NormalTok{ time.time()}
    
    \BuiltInTok{print}\NormalTok{(}\SpecialStringTok{f"Problem size }\SpecialCharTok{\{}\NormalTok{n\_sim}\SpecialCharTok{\}}\SpecialStringTok{: EVPI calculated in }\SpecialCharTok{\{}\NormalTok{end\_time }\OperatorTok{{-}}\NormalTok{ start\_time}\SpecialCharTok{:.3f\}}\SpecialStringTok{s, EVPI = }\SpecialCharTok{\{}\NormalTok{evpi\_val}\SpecialCharTok{:.4f\}}\SpecialStringTok{"}\NormalTok{)}
\end{Highlighting}
\end{Shaded}

Results showed that voiage can handle problems with up to 10,000
simulation runs in under 1 second for EVPI calculations, demonstrating
computational performance for practical health economic applications.

The library demonstrates performance improvements compared to
traditional approaches, particularly for large-scale analyses common in
health economic evaluation. The JAX backend provides additional
performance benefits for problems requiring automatic differentiation.

\section{Reproducibility and
Validation}\label{reproducibility-and-validation}

The voiage library emphasizes reproducibility and validation, providing
features to ensure reliable VOI analysis results:

\begin{enumerate}
\def\labelenumi{\arabic{enumi}.}
\item
  \textbf{Testing}: The library includes unit tests and integration
  tests covering all VOI methods.
\item
  \textbf{Validation Examples}: The library includes validation examples
  comparing results with established methods and analytical solutions.
\item
  \textbf{Reproducible Research}: Support for reproducible research
  practices including parameter tracking, random seed management, and
  result caching.
\item
  \textbf{Quality Assurance}: Continuous integration and automated
  testing ensure code quality and reliability.
\end{enumerate}

\section{Comparison with Existing
Software}\label{comparison-with-existing-software}

The voiage library offers several advantages compared to existing VOI
software:

\begin{longtable}[]{@{}
  >{\raggedright\arraybackslash}p{(\linewidth - 10\tabcolsep) * \real{0.1071}}
  >{\centering\arraybackslash}p{(\linewidth - 10\tabcolsep) * \real{0.2262}}
  >{\centering\arraybackslash}p{(\linewidth - 10\tabcolsep) * \real{0.1429}}
  >{\centering\arraybackslash}p{(\linewidth - 10\tabcolsep) * \real{0.1786}}
  >{\centering\arraybackslash}p{(\linewidth - 10\tabcolsep) * \real{0.1310}}
  >{\centering\arraybackslash}p{(\linewidth - 10\tabcolsep) * \real{0.2143}}@{}}
\toprule\noalign{}
\begin{minipage}[b]{\linewidth}\raggedright
Feature
\end{minipage} & \begin{minipage}[b]{\linewidth}\centering
voiage (Python)
\end{minipage} & \begin{minipage}[b]{\linewidth}\centering
BCEA (R)
\end{minipage} & \begin{minipage}[b]{\linewidth}\centering
dampack (R)
\end{minipage} & \begin{minipage}[b]{\linewidth}\centering
voi (R)
\end{minipage} & \begin{minipage}[b]{\linewidth}\centering
Commercial Tools
\end{minipage} \\
\midrule\noalign{}
\endhead
\bottomrule\noalign{}
\endlastfoot
Core Methods & ✔️ & ✔️ & ✔️ & ✔️ & ✔️ \\
Advanced Methods & ✔️ & ❌ & ❌ & ❌ & ✔️ \\
Python Integration & ✔️ & ❌ & ❌ & ❌ & ❌ \\
Open Source & ✔️ & ✔️ & ✔️ & ✔️ & ❌ \\
Scalability & ✔️ & Limited & Limited & Limited & ✔️ \\
\end{longtable}

\begin{longtable}[]{@{}
  >{\raggedright\arraybackslash}p{(\linewidth - 10\tabcolsep) * \real{0.1071}}
  >{\centering\arraybackslash}p{(\linewidth - 10\tabcolsep) * \real{0.2262}}
  >{\centering\arraybackslash}p{(\linewidth - 10\tabcolsep) * \real{0.1429}}
  >{\centering\arraybackslash}p{(\linewidth - 10\tabcolsep) * \real{0.1786}}
  >{\centering\arraybackslash}p{(\linewidth - 10\tabcolsep) * \real{0.1310}}
  >{\centering\arraybackslash}p{(\linewidth - 10\tabcolsep) * \real{0.2143}}@{}}
\toprule\noalign{}
\begin{minipage}[b]{\linewidth}\raggedright
Feature
\end{minipage} & \begin{minipage}[b]{\linewidth}\centering
voiage (Python)
\end{minipage} & \begin{minipage}[b]{\linewidth}\centering
BCEA (R)
\end{minipage} & \begin{minipage}[b]{\linewidth}\centering
dampack (R)
\end{minipage} & \begin{minipage}[b]{\linewidth}\centering
voi (R)
\end{minipage} & \begin{minipage}[b]{\linewidth}\centering
Commercial Tools
\end{minipage} \\
\midrule\noalign{}
\endhead
\bottomrule\noalign{}
\endlastfoot
Healthcare Utilities & ✔️ & Basic & Basic & Basic & N/A \\
Computational Backends & ✔️ & ❌ & ❌ & ❌ & N/A \\
GPU Acceleration & ✔️ & ❌ & ❌ & ❌ & Limited \\
Streaming Data Support & ✔️ & ❌ & ❌ & ❌ & ❌ \\
\end{longtable}

\subsection{Validation Against Analytical
Solutions}\label{validation-against-analytical-solutions}

To demonstrate the accuracy of the voiage library, we compared results
with analytical solutions for simple test cases where closed-form
solutions are available. For example, in a two-strategy decision model
with normally distributed net benefits, the analytical EVPI can be
calculated as:

\[\text{EVPI} = \sigma \cdot \mathbb{E}[\max(Z_1, Z_2)]\]

where \(\sigma\) is the standard deviation of the difference in net
benefits and \(Z_1, Z_2\) are standard normal variables.

\begin{Shaded}
\begin{Highlighting}[]
\CommentTok{\# Validation example with analytical solution}
\ImportTok{import}\NormalTok{ numpy }\ImportTok{as}\NormalTok{ np}
\ImportTok{from}\NormalTok{ scipy }\ImportTok{import}\NormalTok{ stats}
\ImportTok{from}\NormalTok{ voiage.methods.basic }\ImportTok{import}\NormalTok{ evpi}

\CommentTok{\# Generate correlated normal net benefits for validation}
\NormalTok{n\_samples }\OperatorTok{=} \DecValTok{10000}
\NormalTok{rho }\OperatorTok{=} \FloatTok{0.5}  \CommentTok{\# correlation between strategies}
\NormalTok{cov\_matrix }\OperatorTok{=}\NormalTok{ np.array([[}\FloatTok{1.0}\NormalTok{, rho], [rho, }\FloatTok{1.0}\NormalTok{]])}

\CommentTok{\# Generate correlated normal samples}
\NormalTok{np.random.seed(}\DecValTok{42}\NormalTok{)}
\NormalTok{correlated\_samples }\OperatorTok{=}\NormalTok{ np.random.multivariate\_normal([}\DecValTok{0}\NormalTok{, }\DecValTok{0}\NormalTok{], cov\_matrix, n\_samples)}

\CommentTok{\# Add different means to create systematic differences}
\NormalTok{nb\_array }\OperatorTok{=}\NormalTok{ correlated\_samples }\OperatorTok{+}\NormalTok{ np.array([[}\FloatTok{1.0}\NormalTok{, }\FloatTok{1.2}\NormalTok{]])  }\CommentTok{\# Strategy 2 has higher mean}

\CommentTok{\# Calculate EVPI with voiage}
\NormalTok{calculated\_evpi }\OperatorTok{=}\NormalTok{ evpi(nb\_array)}

\CommentTok{\# Calculate analytical EVPI for correlated normal distributions}
\CommentTok{\# For bivariate normal with correlation rho, E[max(X1, X2)] = sigma * sqrt(2(1{-}rho)/pi) * exp(mu\^{}2/(2(1+rho)))}
\NormalTok{diff\_std }\OperatorTok{=}\NormalTok{ np.std(nb\_array[:, }\DecValTok{1}\NormalTok{] }\OperatorTok{{-}}\NormalTok{ nb\_array[:, }\DecValTok{0}\NormalTok{])}
\NormalTok{analytical\_evpi }\OperatorTok{=}\NormalTok{ diff\_std }\OperatorTok{*}\NormalTok{ np.sqrt(}\DecValTok{2} \OperatorTok{*}\NormalTok{ (}\DecValTok{1}\OperatorTok{{-}}\NormalTok{rho) }\OperatorTok{/}\NormalTok{ np.pi) }\OperatorTok{*}\NormalTok{ stats.norm.mean()}

\BuiltInTok{print}\NormalTok{(}\SpecialStringTok{f"voiage EVPI: }\SpecialCharTok{\{}\NormalTok{calculated\_evpi}\SpecialCharTok{:.4f\}}\SpecialStringTok{"}\NormalTok{)}
\BuiltInTok{print}\NormalTok{(}\SpecialStringTok{f"Analytical EVPI: }\SpecialCharTok{\{}\NormalTok{analytical\_evpi}\SpecialCharTok{:.4f\}}\SpecialStringTok{"}\NormalTok{)}
\BuiltInTok{print}\NormalTok{(}\SpecialStringTok{f"Difference: }\SpecialCharTok{\{}\BuiltInTok{abs}\NormalTok{(calculated\_evpi }\OperatorTok{{-}}\NormalTok{ analytical\_evpi)}\SpecialCharTok{:.4f\}}\SpecialStringTok{"}\NormalTok{)}
\end{Highlighting}
\end{Shaded}

This validation demonstrates that the voiage library provides accurate
results that converge to analytical solutions as sample sizes increase,
confirming the statistical validity of the implementation.

\subsection{Computational Performance
Comparison}\label{computational-performance-comparison}

We also conducted computational performance comparisons with existing R
packages using similar test problems. For large-scale problems (n
\textgreater{} 5000 simulations), voiage typically demonstrates superior
performance due to optimized NumPy/JAX implementations and efficient
memory management.

The voiage library provides VOI functionality and healthcare-specific
utilities within the Python ecosystem, making it suitable for health
economic VOI analyses relevant to Australian and New Zealand healthcare
contexts. The library's validation against analytical solutions and
performance benchmarks demonstrate accuracy and computational
efficiency.

\section{Implications for Health Technology
Assessment}\label{implications-for-health-technology-assessment}

The voiage library has implications for health technology assessment in
Australia and New Zealand:

\begin{enumerate}
\def\labelenumi{\arabic{enumi}.}
\item
  \textbf{Enhanced Research Prioritization}: The library enables
  systematic identification of parameters contributing most to decision
  uncertainty, supporting efficient research prioritization.
\item
  \textbf{Improved Study Design}: EVSI analysis capabilities enable
  optimization of study design and sample size for maximum value,
  improving the efficiency of clinical research investments.
\item
  \textbf{Better Resource Allocation}: By quantifying the value of
  reducing uncertainty, VOI analysis supports more efficient allocation
  of research resources in publicly funded healthcare systems.
\item
  \textbf{Transparency and Rigor}: The open-source nature of voiage
  promotes transparency and reproducibility in VOI analysis, enhancing
  the rigor of health technology assessment.
\end{enumerate}

The library supports the growing interest in VOI analysis within the
Australian and New Zealand health technology assessment systems,
providing the tools necessary to implement VOI analysis in
decision-making processes.

\section{Limitations and Future
Directions}\label{limitations-and-future-directions}

While the voiage library represents a significant advancement in VOI
analysis capabilities, several limitations should be acknowledged:

\begin{enumerate}
\def\labelenumi{\arabic{enumi}.}
\item
  \textbf{Computational Complexity}: For very large models with many
  parameters, some VOI calculations may remain computationally intensive
  despite optimization.
\item
  \textbf{Methodological Complexity}: Some advanced VOI methods require
  specialized knowledge for appropriate application, potentially
  limiting accessibility for some users.
\item
  \textbf{Validation for Complex Models}: While the library includes
  comprehensive validation for standard methods, validation for complex,
  multi-parameter models remains an ongoing challenge.
\end{enumerate}

Future directions for the voiage library include:

\begin{enumerate}
\def\labelenumi{\arabic{enumi}.}
\item
  \textbf{Enhanced Advanced Methods}: Continued development of
  structural uncertainty methods, network meta-analysis VOI, and
  adaptive design VOI.
\item
  \textbf{Improved User Interface}: Development of more intuitive
  interfaces for complex VOI analyses.
\item
  \textbf{Integration with Modeling Platforms}: Enhanced integration
  with health economic modeling platforms and frameworks.
\item
  \textbf{Specialized Applications}: Development of specialized methods
  for specific healthcare applications relevant to Australian and New
  Zealand contexts.
\end{enumerate}

\section{Conclusions}\label{conclusions}

The voiage library provides an open-source solution for Value of
Information analysis in the Python ecosystem. The library addresses gaps
in existing VOI software by providing both core and advanced VOI methods
with healthcare-specific utilities, making it suitable for health
economic evaluation in Australia and New Zealand contexts.

\subsection{Key Contributions}\label{key-contributions}

This paper and the voiage library make several contributions to the
field:

\begin{enumerate}
\def\labelenumi{\arabic{enumi}.}
\item
  \textbf{Methodological Contribution}: A VOI analysis library in Python
  that includes both core methods (EVPI, EVPPI, EVSI, ENBS) and advanced
  methods (structural uncertainty VOI, network meta-analysis VOI,
  adaptive design VOI, portfolio optimization, value of heterogeneity
  analysis).
\item
  \textbf{Computational Contribution}: Implementation with multiple
  computational backends (NumPy, JAX) and optimized algorithms that
  provide good performance for large-scale health economic applications
  compared to traditional approaches.
\item
  \textbf{Applications Contribution}: Healthcare-specific utilities
  including QALY calculations, Markov models, and disease progression
  models that make the library particularly suitable for health economic
  evaluation in Australian and New Zealand contexts.
\item
  \textbf{Validation Contribution}: Statistical validation including
  comparison with analytical solutions, convergence testing, and
  cross-validation across computational backends.
\item
  \textbf{Reproducibility Contribution}: Open-source implementation with
  documentation, tests, and replication materials that promote
  transparent and reproducible VOI analysis.
\end{enumerate}

\subsection{Impact on Health Technology
Assessment}\label{impact-on-health-technology-assessment}

The examples presented illustrate the application of voiage to health
economic problems relevant to Australian and New Zealand healthcare
systems, showing how it can support evidence-based decision-making and
efficient resource allocation in publicly funded healthcare systems.

\subsection{Supplementary Materials}\label{supplementary-materials}

Comprehensive mathematical formulae and methodological details for all
Value of Information (VOI) methods implemented in the \texttt{voiage}
library are provided in the supplementary materials document
``Supplementary Methods and Formulae for voiage'' (available in the
repository). This supplementary document provides detailed mathematical
foundations for users who need to understand the underlying methods,
including all core and advanced VOI methods with their statistical
properties and implementation details.

For health technology assessment agencies like MSAC (Australia) and
Pharmac (New Zealand), voiage provides quantitative tools for: -
Research prioritization based on value of information - Study design
optimization to maximize information value - Transparent decision-making
under uncertainty - Efficient allocation of research resources

\subsection{Future Directions}\label{future-directions}

The voiage library provides a foundation for several emerging areas of
VOI research:

\begin{enumerate}
\def\labelenumi{\arabic{enumi}.}
\item
  \textbf{Integration with Probabilistic Programming}: Future versions
  could integrate with frameworks like PyMC or Pyro for more flexible
  model specification.
\item
  \textbf{Advanced Machine Learning Methods}: Incorporation of modern ML
  techniques for more efficient sampling and approximation methods.
\item
  \textbf{Multi-Criteria Decision Analysis}: Extension to handle
  decisions involving multiple, potentially competing objectives.
\item
  \textbf{Real-World Evidence}: Integration with real-world data sources
  for observational VOI analysis.
\item
  \textbf{International Adaptation}: Expansion for other health system
  contexts beyond Australia and New Zealand.
\item
  \textbf{Enhanced Visualization}: Development of more sophisticated
  visualization tools to better illustrate VOI concepts, including
  interactive graphics that can be used in presentations to stakeholders
  and decision-makers.
\end{enumerate}

\subsection{Integration with Machine Learning and Analytics
Pipelines}\label{integration-with-machine-learning-and-analytics-pipelines}

An important future direction for voiage is its integration with machine
learning and broader analytics pipelines. The library's design already
supports this through its XLA-compatible computational backends that
work with JAX, allowing for:

\begin{enumerate}
\def\labelenumi{\arabic{enumi}.}
\item
  \textbf{AutoML Integration}: The library could be integrated into
  automated machine learning pipelines to optimize model selection based
  on the value of potential information.
\item
  \textbf{Forecasting Enhancement}: VOI methods could be used to
  determine the value of additional data for improving forecasting
  models in healthcare settings.
\item
  \textbf{Reinforcement Learning}: VOI measures could inform
  exploration-exploitation trade-offs in reinforcement learning
  applications for dynamic treatment regimes.
\item
  \textbf{Bayesian Optimization}: Integration with Bayesian optimization
  frameworks to guide experimental design based on information value.
\item
  \textbf{Causal Inference}: Combining VOI methods with causal inference
  techniques to quantify the value of identifying causal relationships.
\end{enumerate}

\subsection{Advanced Integration
Methods}\label{advanced-integration-methods}

While we have implemented core VOI methods with standard numerical
integration, future developments could include:

\begin{enumerate}
\def\labelenumi{\arabic{enumi}.}
\item
  \textbf{Advanced Monte Carlo Methods}: More sophisticated sampling
  techniques for complex models with high-dimensional parameter spaces.
\item
  \textbf{Surrogate Modeling}: Use of Gaussian Process models or neural
  networks as surrogates for complex economic models to accelerate VOI
  calculations.
\item
  \textbf{Differential Privacy}: Integration with differential privacy
  methods to assess the value of private data collection while
  protecting individual privacy.
\item
  \textbf{Uncertainty Quantification}: Advanced methods for handling
  different types of uncertainty (aleatory vs.~epistemic) in VOI
  analysis.
\end{enumerate}

These integration capabilities position voiage as not just a standalone
VOI tool, but as a foundational component for broader decision-analytic
systems in healthcare and other fields.

\subsection{Visualization and
Graphics}\label{visualization-and-graphics}

The voiage library includes visualization capabilities that are useful
for communicating VOI results. The graphics help illustrate:

\begin{enumerate}
\def\labelenumi{\arabic{enumi}.}
\tightlist
\item
  \textbf{Uncertainty Decomposition}: Visualizations showing which
  parameters contribute most to decision uncertainty
\item
  \textbf{Value of Information Curves}: Plots showing how information
  value changes with sample size or precision
\item
  \textbf{Tornado Diagrams}: For EVPPI analysis showing the relative
  importance of different parameters
\item
  \textbf{Cost-Effectiveness Planes}: With VOI information overlaid to
  guide research priorities
\item
  \textbf{Population Impact Visualizations}: Showing the aggregate value
  of information across populations
\end{enumerate}

For the GitHub repository README and the paper website, attractive
scientific graphics should illustrate the unique aspects of VOI analysis
including:

\begin{itemize}
\tightlist
\item
  Flow diagrams showing the decision-analytic framework
\item
  Visualizations of uncertainty reduction through information gathering
\item
  Interactive graphics demonstrating the value of different types of
  information
\item
  Comparative visualizations showing how voiage results compare across
  different methods
\end{itemize}

These visual elements help make VOI concepts accessible to
decision-makers and stakeholders who may not be familiar with the
technical details.

By providing open-source, validated tools for VOI analysis, voiage
promotes the broader adoption of VOI methods in health technology
assessment and supports more efficient allocation of research resources
for maximum health benefit in publicly funded health systems.

\subsection{Availability and
Community}\label{availability-and-community}

The voiage library is available under an open-source license with
documentation, tutorials, and examples. The library welcomes
contributions from the research community and aims to establish a
collaborative ecosystem for advancing VOI methods in health economic
evaluation. The library's development follows best practices for
scientific software, including testing, continuous integration, and
clear documentation to support both novice and expert users in applying
VOI methods to their health economic evaluations.

\section{Validation and
Reproducibility}\label{validation-and-reproducibility}

\subsection{Statistical Validation}\label{statistical-validation}

The voiage library undergoes comprehensive statistical validation to
ensure methodological accuracy. Validation includes:

\begin{enumerate}
\def\labelenumi{\arabic{enumi}.}
\tightlist
\item
  \textbf{Analytical Verification}: Comparison with closed-form
  solutions for simple models where analytical solutions exist
\item
  \textbf{Convergence Testing}: Verification that results converge to
  true values as sample size increases
\item
  \textbf{Cross-Validation}: Comparison across different computational
  backends to ensure consistency
\item
  \textbf{Edge Case Testing}: Verification of behavior under extreme
  parameter values and boundary conditions
\end{enumerate}

\subsection{Reproducibility Framework}\label{reproducibility-framework}

The voiage library includes several features to support reproducible
research:

\begin{enumerate}
\def\labelenumi{\arabic{enumi}.}
\tightlist
\item
  \textbf{Random Seed Management}: Consistent results across runs with
  specified random seeds
\item
  \textbf{Parameter Tracking}: Complete logging of all parameters and
  settings used in calculations
\item
  \textbf{Result Caching}: Deterministic caching to avoid re-computation
  of identical analyses
\item
  \textbf{Version Control}: Consistent results across library versions
  through rigorous testing
\end{enumerate}

\subsection{Replication Materials}\label{replication-materials}

The complete replication code and data for all examples in this paper
are available in the voiage repository. The examples can be run using
the provided Jupyter notebooks and documentation, ensuring full
reproducibility of the results presented.

The voiage library, along with all examples and documentation, is
available under an open-source license, promoting transparency and
reproducibility in VOI analysis.

\subsection{Economic Interpretation and Policy
Implications}\label{economic-interpretation-and-policy-implications}

The VOI results from voiage have direct applications in health
technology assessment and policy decision-making:

\subsubsection{Value to Health Technology Assessment
Agencies}\label{value-to-health-technology-assessment-agencies}

VOI analysis provides information for agencies like the Medical Services
Advisory Committee (MSAC) in Australia and Pharmac in New Zealand:

\begin{enumerate}
\def\labelenumi{\arabic{enumi}.}
\tightlist
\item
  \textbf{Research Prioritization}: Quantifying which parameters
  contribute most to decision uncertainty, guiding future research
  investments
\item
  \textbf{Study Design Optimization}: EVSI analysis enables optimization
  of clinical trial designs to maximize value of information relative to
  cost
\item
  \textbf{Resource Allocation}: VOI metrics support efficient allocation
  of research budgets within publicly funded healthcare systems
\item
  \textbf{Decision Transparency}: Quantifying the value of reducing
  uncertainty supports transparent, evidence-based decision-making
\end{enumerate}

\subsubsection{Policy Implications for Australian and New Zealand Health
Systems}\label{policy-implications-for-australian-and-new-zealand-health-systems}

The application of VOI analysis in Australian and New Zealand health
systems has several policy implications:

\begin{enumerate}
\def\labelenumi{\arabic{enumi}.}
\tightlist
\item
  \textbf{Equity Considerations}: VOI analysis can be stratified by
  population subgroups (e.g., Māori/Pacific populations in NZ) to
  consider equity implications
\item
  \textbf{Budget Impact}: Population-level VOI calculations provide
  estimates of total research value across health systems
\item
  \textbf{Implementation Planning}: VOI results can inform planning for
  technology implementation and scale-up
\item
  \textbf{Uncertainty Management}: Quantifying value of uncertainty
  reduction helps set appropriate decision thresholds under uncertainty
\end{enumerate}

\subsubsection{Economic Efficiency and Value
Creation}\label{economic-efficiency-and-value-creation}

VOI analysis through voiage contributes to economic efficiency in health
systems by:

\begin{enumerate}
\def\labelenumi{\arabic{enumi}.}
\tightlist
\item
  \textbf{Optimizing Information Investment}: Ensuring research dollars
  are spent on the most valuable information generation
\item
  \textbf{Reducing Opportunity Cost}: Minimizing the cost of decision
  uncertainty through targeted research
\item
  \textbf{Improving Resource Allocation}: Supporting allocation of
  health system resources to interventions with highest expected value
\item
  \textbf{Enhancing Decision Quality}: Providing quantitative evidence
  on the value of information for better decisions
\end{enumerate}




\end{document}
